\documentclass[12pt,a4paper]{article}
\usepackage[utf8]{vietnam}
\usepackage{amsmath}
\usepackage{amsfonts}
\usepackage{amssymb}
\usepackage{graphicx}
\usepackage[left=2cm,right=2cm,top=2cm,bottom=2cm]{geometry}
\begin{document}
\textbf{2.1.2. Nhiệm vụ lịch sử của thời kì quá độ lên chủ nghĩa xã hội ở Việt Nam.}\\Thực chất thời kì quá độ lên chủ nghĩa xã hội Việt Nam là cải tiến nền sản xuất lạc hậu thành nề sản xuất tiên tiến, hiện đại.\\
Nhiệm vụ của thời kì quá độ lên chủ nghĩa xã hội Việt Nam bao gồm hai nội dung lớn:\\
- Xây dựng nền tảng vật chất và kĩ thuật, xây dựng các tiền đề kinh tế, chính trị, văn hóa, tư tưởng.\\
- Cải tạo xã hội cũ, xây dựng xã hội mới, trong đó xây dựng làm trọng tâm, làm nội dung cốt lõi nhất, chủ chốt và lâu dài.\\
Những khó khăn trong quá trình thực hiện:\\
- Đây thực sự là một cuộc cách mạng làm đảo lộn mọi mặt đời sống xã hội, cả lực lượng sản xuất và quan hệ sản xuất, cả cơ sở hạ tầng và kiến trúc thượng tầng.\\
- Đảng, Nhà nước, và nhân dân ta chưa có kinh nghiệm, nhất là trên lĩnh vực kinh tế. \\
- Các thế lực phản động trong và ngoài nước tìm cách chống phá.\\[0.5cm]
\textbf{2.1.3. Quan điểm của hồ Chí Minh về nội dung xây dựng chủ nghĩa xã hội ở nước ta trong thời kì quá độ.}\\
\textbf{a) Trong lĩnh vực chính trị}\\
Công cuộc xây dựng chủ nghĩa xã hội ở nước ta là sự nghiệp cách mạng mang tính toàn diện. Hồ Chí Minh đã xác định rõ nhiệm vụ cụ thể cho từng lĩnh vực:\\[0.5cm]
- Trong lĩnh vực chính trị, nội dung quan trọng nhất là phải giữ vững và phát huy vai trò lãnh đạo của Đảng. Đảng phải luôn luôn tự đổi mới và tự chỉnh đốn, nâng cao năng lực lãnh đạo và sức chiến đấu, có hình thức tổ chức phù hợp để đáp ứng các yêu cầu, nhiệm vụ mới. Bước vào thời kỳ quá độ lên chủ nghĩa xã hội, Đảng ta đã trở thành Đảng cầm quyền Mối quan tâm lớn nhất của Người về Đảng cầm quyền là làm sao cho Đảng không trở thành Đảng quan liêu, xa dân, thoái hóa, biến chất, làm mất lòng tin của dân, có thể dẫn đến nguy cơ sai lầm về đường lối, cắt đứt mối quan hệ máu thịt với nhân dân và để cho chủ nghĩa cá nhân nảy nở dưới nhiều hình thức.\\
Đồng thời, củng cố  và tăng cường vai trò quản lý của nhà nước trong sự nghiệp xây dựng chủ nghĩa xã hội ngày càng trở thành nhiệm vụ rất quan trọng.\\
Một nội dung chính trị quan trọng trong thời kỳ quá độ lên chủ nghĩa xã hội là củng cố và mở rộng mặt trận dân tộc thống nhất, nòng cốt là liên minh công nhân, nông dân và tri thức, do Đảng Cộng sản lãnh đạo; củng cố và tăng cường sức mạnh toàn bộ hệ thống chính trị cũng như thành tố của nó.\\[0.5cm]
\textbf{b) Trong lĩnh vực kinh tế}\\
- Nội dung kinh tế được Hồ Chí Minh đề cập trên các mặt: lực lượng sản xuất, quan hệ sản xuất, cơ chế quản lý kinh tế. Người nhấn mạnh đến việc tăng năng suất lao động trên cơ sở tiến hành công nghiệp hóa xã hội chủ nghĩa. Đối với cơ cấu kinh tế Hồ Chí Minh đề cập cơ cấu ngành và cơ cấu các thành phần kinh tế, cơ cấu kinh tế vùng lãnh thổ.\\
Người quan niệm hết sức độc đáo về cơ cấu kinh tế nông - công nghiệp, lấy nông nghiệp làm mặt trận hàng đầu, củng cố hệ thống thương nghiệp làm cầu nối tốt nhất giữa các ngành sản xuất xã hội, thỏa mãn nhu cầu thiết yếu của nhân dân.\\
Đối với kinh tế vùng, lãnh thổ. Hồ Chí Minh lưu ý phải phát triển đồng đều giữa kinh tế đô thị và kinh tế nông thôn. Người đặc biệt chú trọng chỉ đạo phát triển kinh tế vùng núi, hải đảo, vừa tạo điều kiện không ngừng cải thiện và nâng cao đời sống của đồng bào, vừa bảo đảm an ninh, quốc phòng cho đất nước.\\
Ở nước ta, Hồ Chí Minh là người đầu tiên chủ trương phát triển cơ cấu kinh tế nhiều thành phần trong suốt thời kỳ quá độ lên chủ nghĩa xã hội. Người xác định rõ vị trí và xu hướng vận động của từng thành phần kinh tế. Nước ta cần ưu tiên phát triền kinh tế quốc doanh để tạo nền tảng vật chất cho chủ nghĩa xã hội, thúc đẩy việc cải tạo xã hội chủ nghĩa. Kinh tế hợp tác xã là hình thức sở hữu tập thế của nhân dân lao động, Nhà nước cần đặc biệt khuyến khích, hướng dẫn và giúp đỡ nó phát triển, về tổ chức hợp tác xã, Hồ Chí Minh nhấn mạnh nguyên tắc dần dần, từ thấp đến cao, tự nguyện, cùng có lợi, chống chủ quan, gò ép, hình thức. Đối với người làm nghề thủ công và lao động riêng lẻ khác. Nhà nước bao hộ quyền sở hữu về tư liệu sản xuất, ra sức hướng dẫn và giúp họ cải tiến cách làm ăn. khuyến khích họ đi vào con đường hợp tác. Đối với những nhà tư sản công thương, vì họ đã tham gia ủng hộ cách mạng dân tộc dân chủ nhân dân, có đóng góp nhất định trong khôi phục kinh té và sẵn sàng tiếp thu, cải tạo đế góp phần xây dựng nước nhà, xây dựng chủ nghĩa xã hội, nên Nhà nước không xóa bỏ quyền sở hữu về tư liệu sản xuất và của cải khác của họ, mà hướng dẫn họ hoạt động làm lợi cho quốc kế dân sinh, phù hợp với kinh tế nhà nước, khuyến khích và giúp đỡ họ cải tạo theo chủ nghĩa xã hội bằng hình thức tư bản nhà nước.\\
Bên cạnh chế độ và quan hệ sở hữu. Hồ Chí Minh rất coi trọng quan hệ phân phối và quản lý kinh tế. Quản lý kinh tế phải dựa trên cơ sở hạch toán, đem lại hiệu quả cao, sử dụng tốt các đòn bẩy trong phát triển sản xuất Người chủ trương và chi rõ các điều kiện thực hiện nguyên tắc phân phối theo lao động: làm nhiều hưởng nhiều, làm ít hưởng ít, không làm không hưởng. Gắn liền với nguyên tắc phân phối theo lao động, Hồ Chí Minh bước đầu đề cập vấn đề khoán trong sản xuất, "Chế độ làm khoán là một điều kiện của chủ nghĩa xã hội. nó khuyến khích người công nhân luôn luôn tiến bộ, cho nhà máy tiến bộ. Làm khoán là ích chung và là lợi riêng...: làm khoán tốt thích hợp và công bằng dưới chế độ ta hiện nay\\[0.5cm]
\textbf{c) Trong lĩnh vực văn hóa xã hội}\\
- Trong lĩnh vực văn hóa - xã hội, Hồ Chí Minh nhấn mạnh đến vấn đề xây dựng con người mới. Đặc biệt, Hồ Chí Minh đề cao vai trò của văn hóa, giáo dục và khoa học – kỹ thuật trong xã hội xã hội chủ nghĩa. Người cho rằng, muốn xây dựng chủ nghĩa xã hội nhất định phải có học thức, cần phải học cả văn hóa, chính trị, kỹ thuật và chủ nghĩa xã hội cộng với khoa học chắc chắn đưa loài người đến hạnh phúc vô tận. Hồ Chí Minh rất coi trọng việc nâng cao dan trí, đào tạo và sử dụng nhân tai, khẳng định vai trò to lớn của văn hóa trong đời sống xã hội.



\end{document}